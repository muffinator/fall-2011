\documentclass{article}
\usepackage{graphicx}
\usepackage{amsmath}
\author{Josh Gordonson}
\title{6.331 Pset 2}

\begin{document}
\maketitle{}
\section{}
\subsection{}
%1a
The maximum slew rate of the hold capacitor can be calculated from $\frac{I_o}{C_h}$
\begin{align}
Slew_{max+} & = \frac{10mA}{100pF} = 10^8 [\frac{V}{s}] \label{slw}\\
Slew_{max-} & = \frac{9mA}{100pF} = 10^7 [\frac{V}{s}] 
\end{align}
The error caused by the given current imbalance ($10mA$ on top, $9mA$ on bottom) can 
be found by modeling the incremental resistance of the diodes at their respective 
quiescent currents, assuming a $v_D$ of $V_{TH} = 25mV$:
\begin{align}
r_D &= \frac{v_D}{i_D} = \frac{2V_{TH}}{I_D} \\
r_{D_{top}} &= \frac{2KT}{5mA} = 10 \Omega \label{incres}\\
r_{D_{bot}} &= \frac{2KT}{4.5mA} = 11.1 \Omega 
\end{align}
Now, we use the incremental resistive model to find the incremental error caused by 
current mismatch in the two legs of the floating current source. To do this, we find 
the equivalent resistance of one of the top legs in parallel with the rest of the legs in series.
$$
\frac{10 \Omega (2*11.1 \Omega+10 \Omega)}{20 \Omega+2*11.1 \Omega}1mA = 6mV
$$
\subsection{}
%1b
From \eqref{slw}:
$$
\Delta V_{Slew} = \frac{I_o}{C_{Hold}} = 10^8 \frac{V}{s}, \qquad \frac{10}{10^8 \frac{V}{s}}=100nS
$$
The change in voltage due to the switching lag can be calculated by multiplying the slew rate by
$\Delta t$.
$$
\Delta V_{error} = \frac{I_o \Delta t}{2C_{hold}} = \frac{10^8*10^{-9}}{2} = 50mV
$$
\subsection{}
To evaluate the effect of stray capacitance on each diode, we look at the equillibrium case 
where the input and output are held at 5V.  At switching time, the diodes are reverse-biased,
causing charge dump proportional to the revere-bias voltage.  On the input half of the diode
network, the charge is supplied by the input signal.  On the output half of the diode network,
 the charge is sourced from $C_{hold}$.  \\
\\
First, we find $\Delta V_{D_{top}}$, which is equal to the change in voltage across the top diode;
 $\Delta V_{D_{top}} = .6-20=-20.6V$.  Next, you can find the amount of charge that is dumped due
to the abrupt change in $V_D$ by using the relationship $Q=CV$; $\Delta Q_{d_{top}}=C_{top}\Delta
 V_{D_{top}}=-.206nC$.  Rinse and repeat for the bottom diode; $\Delta V_{D_{bot}}=10.6V \qquad 
\Delta Q_{d_{bot}}=.106nC$. \\
\\
Now you have to find the change in charge on $C_{hold}$.
\begin{align}
\Delta Q_{hold} &=\Delta Q_{d_{top}} + \Delta Q_{d_{bot}}=-.10nC \\
\Delta V_{out} &= \frac{\Delta Q_{hold}}{C_{hold}}=-1V
\end{align}
\subsection{}
To find the step response of the sample and hold in sample mode you need an incremental model of the diode network, replacing each diode with a resistor of value $r_d$ equal to its incremental resistance.  From there, the problem simplifies to a first order low-pass filter. From equation \eqref{incres}:
\begin{align}
r_d &= 10\Omega\\
\frac{V_o}{V_i} &= \frac{1}{r_dc_hs+1}\\
\frac{V_o}{V_i} \frac{1}{s} &= \frac{1}{r_dc_hs^2+s}\\
 &= 1-e^{\frac{-t}{r_dc_h}}\\
.999 &= 1-e^{\frac{-t}{r_dc_h}}\\
 t_{0.1\%} &= -r_dc_hln(.001)\\
\int_0^t{1-e^{\frac{-t}{r_dc_h}}dt} &= t+r_dc_he^{\frac{-t}{r_dc_h}}-r_dc_h
\end{align}
\section{}
The real component of the incremental input impedance is given by the following transfer
 function, which follows from using KCL at the emitter node with a test current on the base.
\begin{align}
I_{test} + I_c &= I_e, \qquad I_{test}+I_c = V_\pi\frac{1}{r_\pi\|c_\pi} + g_mV_\pi \\
V_\pi\frac{1}{r_\pi\|c_\pi} + g_mV_\pi &= V_eC_Es ,\qquad  V_e = V_{test}-V_\pi \\
V_{test} &= I_{test}r_\pi\|c_\pi(\frac{\frac{1}{r_\pi\|c_\pi}+g_m}{C_Es}+1) \\
Z_{In} = \frac{V_{test}}{I_{test}} &= 
	\frac{1+g_m r_\pi \| c_\pi}{C_E s} + r_\pi \| c_\pi, \qquad
	r_\pi \| c_\pi = \frac{r_\pi}{r_\pi c_\pi s + 1} \\
 &= \frac{r_\pi c_\pi s+1+g_m r_\pi}{C_E s(r_\pi c_\pi s +1)} +
	\frac{r_\pi}{r_\pi c_\pi s + 1} \\
 &= \frac{r_\pi(c_\pi s+C_E s+ g_m)+1}{C_E s(r_\pi c_\pi s+1)}
	\frac{-C_E r_\pi c_\pi w^2 - C_Ejw}
	{-C_E r_\pi c_\pi w^2 - C_Ejw} \label{plot}\\
\Re \{Z_{In}\} &= \frac{(C_E - g_m r_\pi c_\pi)r_\pi}
	{r_\pi^2c_\pi^2C_Ew^2+C_E} 
\end{align}
So, If $C_E < g_m r_\pi c_\pi$, then the incremental input resistance to the amplifier is negative for all frequencies. \\
To plot the incremental input impedance, we use equation \eqref{plot} \\
$$
Z_{In} = \displaystyle\frac{\displaystyle\frac{r_\pi(c_\pi +C_E)}{r_\pi g_m + 1} s+1}
	{\displaystyle\frac{C_E}{r_\pi g_m +1} s(r_\pi c_\pi s+1)}
$$
Clearly, we have a zero at $w=\frac{r_\pi(c_\pi +C_E)}{r_\pi g_m + 1} s+1$,
 an integrator with gain $\frac{r_\pi g_m +1}{C_E} $, and a pole at $w=\frac{1}{r_\pi c_\pi}$ 

\section{}
\end{document}

